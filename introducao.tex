%!TEX root = index.tex
\chapter[Introdução]{Introdução}
\label{chap:introducao}
\section{Contextualização do trabalho}
\label{cha:contexto}

Este trabalho mostra a contextualização do Brasil de encontro com a corrupção dos dias atuais. Sendo essa a visão o objetivo da autora que se baseou para realizar o trabalho de conclusão de curso estava inserida. 

\section{Definição do problema}
\label{cha:definição_do_problema}
O Brasil esta envolvido em uma serie de escandalos de corrupção, sendo visto como um país de enorme extenção territorial porém fadado ao fracasso, ja que a preocupação dos seus representantes com a população que o elegeu é nula ou muito proximo a isso. 

Alem disso, ainda que existam diversas leis, o jeitinho do brasileiro sempre se sobrepoe ao seu dever civil e temos resultados desastrosos que desaguam em ações e operações como a Lava Jato para que haja uma barragem na lavagem de dinheiro e na corrupção intrinsica nas empresas e nos corpos daqueles que seguem com isso. 


\section[Objetivos]{Objetivos}
\label{chap:objetivos}
O Objetivo desse trabalho é mostrar de forma abrangente o estudo da Lei Anticorrupção (Lei n. 12.846/2013), e comparar sua importancia antes e depois da Operação lava jato, problematizando a reestruturação se utilzando de um estudo de caso focado na JBL. 

\section{Justificativa}
\label{cha:justificativa}
Para a autora a importância do tema desse trabalho é imensa dado que a situação permanece latente e a discussão é recente. Por tanto, uma compilação de informações e um estudo de caso leva o conhecimento a demais pessoas que podem estar vivendo em situações parecidas ainda que em empresas menores. Esse trabalho e a oportunidade de difundir conhecimento de forma abrangente dentro de um curso de Direito já que essa meteria não esta diretamente na grade curricular do curso. 

Para a sociedade a importância é mostrar que a Lei Anticorrupção trouxe uma mudança significativa na estruturação de empresas de grande porte, o que levara influencia direta sobre empresas de pequeno e medio porte, causando uma mudança estrutural nas pessoas e possivelmente na sociedade como um todo. 

\section{Estrutura do Trabalho}
\label{cha:estrutura_do_trabalho}
O trabalho de formatura foi estruturado conforme apresentado abaixo:

No Capítulo 1 a autora introduz o contexto historico do Brasil no momento da criação do trabalho (2019). Apresenta a estrutura do trabalho, sua motivação e a importancia da produção desta dissertação para difussão de conhecimento e compilado de informações relevantes sobre o tema. 

O Capítulo 2 aprofunda informações sobre a Lei Anticorrupção e o significado do compliance.

No Capítulo 3 a autora mostra como funcionada o Compliance em meados de 2015 e como esse instituto estava estruturado dentro das empresas, quais eram suas funções e qual a sua relevancia. 

O Capítulo 4 apresenta a relevancia do compliance atualmente e como é aplicado na empresa JBL utilizada como estudo de caso para refletir a atual importancia do compliance e quais suas funcoes e instauração dentro da Cia. 

O Capítulo 5 finaliza o trabalho com a problematização do compliance atual e a resolução da importancia da criação da Lei Anticorrupção. Finalizando as ideias da Autora. 

Por fim, a conclusão, amarrando todas as pontas desse trabalho com uma sintese breve do trabalho e de sua problematização e resolução. 

