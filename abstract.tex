%!TEX root = index.tex
% resumo em português
\setlength{\absparsep}{18pt} % ajusta o espaçamento dos parágrafos do resumo
\begin{resumo}[Abstract]
 \begin{otherlanguage*}{english}
The purpose of this undergraduate thesis of the Industrial Engineering course of the Polytechnic School of the University of São Paulo hereby presented was to restructure a startup that has had a turbulent start. The method that was applied was based on the principles of the Lean Startup, the Customer Development and the Business Model Canvas.

The startup of this thesis developed a discount app for brick-and-mortar shops, whose difference was the use of beacons technology to achieve better accuracy for geolocation indoors. Thus, the app would offer the best discount for a user based on the location of the same. The main challenge was to grow the company on both the user and the brick-and-mortar side.

The author of this thesis is one of the founders of the company and was present throughout the decision-making processes and application development. For the work to be developed the author elaborated a method to make it possible to test systematically the key assumptions of the company. In addition, the method demanded that the learning acquired during each test cycle were recorded for the company not to commit the same mistakes again.

In the first test cycle the goal was to test whether users saw value in the application, if it was possible to get new users inexpensively using social networks and if the retailers were interested in joining this new application.

In the second test cycle the goal was to see if retailers were willing to pay to use the service, as to date no partner was paying for the use of the platform provided by the application, ie, the company's revenue was nil.

The presented work is the result of a search to try to save a company, whose dream was to revolutionize the Brazilian physical retail and one day the world.

   \vspace{\onelineskip}
 
   \noindent 
   \textbf{Keywords}: Beacons, Lean, Startup, Retail.
 \end{otherlanguage*}
\end{resumo}
