%!TEX root = index.tex
% resumo em português
\setlength{\absparsep}{18pt} % ajusta o espaçamento dos parágrafos do resumo
\begin{resumo}[Abstract]
 \begin{otherlanguage*}{english}

  
  This is a work of undergraduate conclusion of the Faculty of Law of São Bernardo do Campo, written by Aluna Bruna Guglielmi Pereira in 2019.

The main point of this paper is firstly to present the compliance / business integrity program, discussing the Law that gave rise to this practice, Law No. 12.846 / 2013 (Anti-Corruption Law), as well as a brief presentation of Decree No. 8.420 / 2015. The Anti-Corruption Law was inspired by major international norms that thanks to our legislators today apply in Brazil to end abusive and corrupt practices in our particular system between companies and the administrative public regulating this interaction.

Thus, the structure of compliance is highlighted some of its nuances and discussed at the end what are the practical problems of its application that we can not find solution only with the use of regulatory standards, doctrines and principles, based on journalistic articles on large cases of the present.


   \vspace{\onelineskip}
 
   \noindent 
   \textbf{Keywords}:
   Compliance, Business Law, Law nº 12.846/2013, Anti-Corruption Law.
 \end{otherlanguage*}
\end{resumo}
