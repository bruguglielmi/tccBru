%!TEX root = index.tex
\chapter{Compliance Antes da Lava Jato}
Conforme apresentado anteriormente, o produto da empresa Beeconnect é o aplicativo com o mesmo o nome. Tal produto foi desenvolvido para as plataformas Android e iOS e consiste basicamente em um aplicativo para descontos em lojas físicas. O seu principal diferencial é a geolocalização indoor precisa com uso de um aparelho chamado beacon. Quando um usuário do aplicativo passar por um beacon localizado dentro de uma loja parceira ele pode receber uma notificação informando que ele recebeu um desconto especial em um produto relevante ou receber um simples \enquote{Bem vindo} conforme mostrado.

Até o momento em que os testes foram realizados a Beeconnect contava com cerca de 2000 downloads do aplicativo, 1600 usuários cadastrados e 10 empresas parceiras. Um dos problemas é que nenhuma loja ainda estava disposta a pagar pela plataforma.

Como mencionado previamente tais desafios eram realmente difíceis de serem resolvidos porque muitos usuários só baixariam o aplicativo se ele possuísse mais lojas participantes, assim como muitas lojas só se interessavam pela base de usuários e só entrariam no aplicativo caso a base fosse grande, com mais de cem mil usuários.

Baseando-se na metodologia apresentada no capítulo anterior o autor então começou a desenvolver os testes e resultados que serão apresentados nesse capítulo. Foram feitas duas iterações no ciclo explicitado pela metodologia introduzida no capítulo anteriror.

\chapter{Complince - JBL pós Lava Jato }



\chapter{Problematização e Resultado}



