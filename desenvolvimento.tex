%!TEX root = index.tex
\chapter{A Lei n. 12.846/ 2013}

Trataremos, primeiramente, da base sobre o instituto de Compliance, a Lei n. 12.846 - sera tratada como LAC - , criada em 1 de agosto de 2013. Esta Lei é denominada como "Lei Anticorrupção"ou "Lei da Empresa Limpa". Conforme seu artigo 1, parágrafo único, aplica-se a todas as sociedades empresariais ou sociedades simples, ou seja, personificadas ou não, independente de fatores como sua organização ou modelo societário adotado. Esse foi um significante marco de enfrentamento da corrupção, especialmente ao estabelecer limites a responsabilização objetiva, civil e administrativa, da pessoa jurídica em razão de atos praticados contra a Administração Publica Estatal, nacional ou estrangeira. 

A norma analisada pune, também, na modalidade subjetiva, os dirigentes e administradores da pessoa jurídica infratora (artigo 3, §2 da LAC), isso porque, acredita-se que as pessoas de alto escalão dentro da empresa deve ter conhecimento sobre tudo que ali acontece, tendo consciência de operações que visam a corrupção para beneficio próprio. Além disso, a Lei não isenta a responsabilização da pessoa jurídica caso o ato seja praticado por terceiro (art. 2 da LAC), tampouco caso haja alteração contratual, transformação ou incorporação, cisão ou fusão societária (art. 4 da LAC), estabelecendo, assim, regras e limites claros para que haja a devida punição objetiva da pessoa jurídica envolvida no esquema de corrupção. 

Alem disto, à previsão para responsabilização objetiva da pessoa jurídica, trazendo o legislador inovação ao prever que para a aplicação das sanções, haverá a avaliação quanto ao eventual programa de integridade no âmbito da pessoa jurídica, definindo na norma como "mecanismos e procedimentos internos de integridade, auditoria e a aplicação efetiva de códigos de ética e conduta" (Art. 7, VIII da LAC)

Assim, na medida em que o legislador estabelece a responsabilidade objetiva contra a pessoa jurídica, a qual o prescinde da avaliação de dolo ou culpa na ocorrência do ato lesivo, a norma indica também o programa de integridade como um instrumento a favor da pessoa jurídica, dando mais importância ao departamento do compliance, que esta totalmente ligado com o programa de integridade que deve existir nas sociedades simples ou empresariais, conforme será demonstrado. Diante da mesma Lei, seja no enfrentamento dos atos de corrupção em suas atividades empresariais e, portanto, de impedimento da incidência da Lei Anticorrupção, seja na eventual fase de dosimetria de sua sanção. 

As características previstas nesta lei, sobre o programa de integridade da pessoa jurídica foi diretamente influenciada pelo contexto jurídico internacional dos Estados Unidos da América, cujos reflexos culminaram na celebração de relevantes acordos internacionais de enfrentamento da corrupção, assinadas e internalizadas pelo Brasil. 

Neste sentido , podemos afirmar que a Lei Anticorrupção é o eficaz produto de pelo menos três convenções internacionais de que o Brasil assinou, e se comprometeu a dar eficacia, sendo eles os atos da FCPA - Foreign Corrupt Practicies Act, a UKBA - United Kingdom Bribery Act, e por fim a OCDE - Convenção sobre o Combate da Corrupção de Funcionários Públicos Estrangeiros em Transações Comerciais Internacionais. A consonância com o contexto internacional da Lei Anticorrupção, pode representar um grande marco jurídico, ainda mais por incentivar e reafirmar os padrões da ética empresarial, internamente e com as relações negociais perante a Administração Pública Estatal. 

\section{Programa de Integridade}

O programa de Integridade, presente na Lei n. 12.846/2013 em seu art. 7, VIII, foi implementado para trazer novas perspectivas as sociedades empresariais de modo genérico.  O programa deve se ater a três níveis de relações: (i) relações internas; (ii) relações externas com outras pessoas jurídicas; e por fim, (iii) relações externas com o Estado.

As empresas precisam, a partir da vigência dessa Lei, de um canal que poderá ser utilizados por todos os empregados para realizarem denuncias de comportamentos corruptos em todos os níveis hierárquicos internos e externos. Além disso, é necessário que haja um grupo de pessoas que não se submeta a hierarquia dentro da pessoa jurídica, para que eles tenham plenos poderes de investigação e combate a praticas abusivas e corruptas, vindo de qualquer pessoa, seja qual for seu grau de poder dentro do desenvolvimento das atividades empresariais, acusando e detendo, portanto, quaisquer praticas consideradas abusivas perante esta norma. 

Ademais, como a pessoa jurídica nada mais é que uma entidade formada por na verdade, diversas pessoas físicas que ali atuam e presam por essa entidade em que trabalham, a Lei Anticorrupção prevê que sejam criadas medidas e procedimentos que garantem a possibilidade de identificação, denuncia e aplicação de métodos de coibição mediante os atos de corrupção repudiados pelas politicas da empresa e pela lei em questão. 

Como meio de complemento a Lei Anticorrupção, o Decreto n. 8.420/2015, nos seus artigos 41 e 42, exemplifica itens que devem fazer parte do programa de integridade para que ele seja valido. Como mera exemplificação, o inciso primeiro do artigo 40 deste decreto vale destaque: "I- Comprometimento da alta direção da pessoa jurídica, incluindo os conselhos, evidenciado pelo apoio visível e inequívoco do programa". Ou seja, como breve explicação, as pessoas com mais poder na empresa devem mostrar a todos os funcionários que elas também estão sujeitas ao programa e todos devem respeita-lo. 

Este instituto traz consigo uma carga de preocupação quanto a sua pratica real. Ainda que haja expressa previsão da retirada desse programa de integridade da escala hierárquica da empresa, é certo a um órgão de alto escalão empresarial não queira e nao goste da ideia de se submeter a outro órgão que nao aqueles expressamente superiores. Como isso, ainda que a norma e o decreto, em forma conjunta, estabeleçam um grande ponto de partida ao combate da corrupção, temos um problema pratico e real quanto a eficiência desta medida. 

Ainda assim, a obrigação das sociedades de pequeno ou grande porte é de criar um canal onde seus funcionários possam fazer denuncias de irregularidades. Ademais, as pessoas que gerenciam esse canal devem ter poder para investigar essas denuncias a fundo para fazer com que elas cessem se confirmadas. Sendo assim, observado matérias, vg. \cite{Gazetaonline}, sobre grandes empresas, como Petrobras, Odebrecht e JBS, esse Programa de Integridade é atualmente um órgão/departamento, que trata da area de compliance da empresa, portanto, o programa nada mais é que o compliance aplicado nas empresas instaladas no Brasil. 

Veja que a ligação é direta, a importância do Compliance Empresarial diante a Lei Anticorrupção de 2013 é clara, sendo que esta prevê a criação desse canal, que foi aplicada com a criação de um departamento exclusivo para tratar desses assuntos e analise de níveis de correlações internas e externas. O departamento funciona como um comité de ética, com independência em relação a todos os outros órgãos da pessoa jurídica. 

O departamento de Compliance surgiu, portanto, da necessidade da fiscalização geral da empresa, da necessidade da criação de um canal aberto, onde todos se submetem e podem fazer suas denuncias de forma segura, na tentativa da criação de uma empresa solida e sem abusos de poderes e praticas corruptas de qualquer natureza. Assim, devido a complexidade desse órgão, devido as suas múltiplas funções, a criação e implementação de um departamento separado se faz necessária e essencial para essas pessoas jurídicas. 


\section{Das Convenções Internacionais assinadas pelo Brasil contra as praticas de corrupção empresarial.}

Como forma de breve esclarecimento, a titulo da menção feita acima, esclarece-se o que são as convenções assinadas pelo Brasil na tentativa de combater e dar mais força a sua repudia como Estado sobre empresas que se utilizam de praticas abusivas e ilegais para obtenção de lucro ilícito, através de praticas como a lavagem de dinheiro. 

O FCPA (Foreign Corrupt Practices Act), foi promulgado pelo Congresso Norte-Americano em 1977, também conhecida como "Lei Contra Praticas de Corrupção Estrangeira". Esse foi um momento marcante para enfrentar a corrupção internacional e corporativista dos Estados Unidos da America em relação aos outros países. Essa norma começou a responsabilizar civil e criminalmente todas as pessoas físicas e jurídicas que cometem suborno ao poder público estrangeiro. Proíbe também, de maneira expressa, a "contabilidade off-the-books", que significa a proibição de omissão de valores e recursos utilizados dos livros de registros corporativos da pessoa jurídica.

Após a norma ser assinada pelo Congresso, devido a grande turbulência no cenário americano com diversas explosões de casos como "Watergate" e "Lockheed"\footnote{Passando brevemente sobre a contextualização de casa um dos casos sitados, primeiro, o caso Lockheed diz respeito a uma empresa multinacional aeronáutica dos E.U.A, que pagou milhões de dólares pra funcionários públicos de outros países para que se utilizassem de suas aeronaves. Já o caso Watergate, em uma situação levemente diferente, possuia diversos prédios e se utilizou de um desses para instalar diversos profissionais que estavam se utilizando de escutas para ter total acesso da situação as estrategias de campanha presidencial dos E.U.A}, as investigações se aprofundaram e foi identificada um rede de corrupção global, forçando os Estados Unidos a transformar tal norma em um Tratado Internacional que foi assinado pelos membros da OCDE (Organização para a Cooperação e Desenvolvimento Econômico)\cite{CriminalDivisionofUS}. \footnote{UNITED STATES OF AMERICA. Criminal Division of the U.S Department of Justice; Enforcement Division of the U.S Secretiries and Exchange Comission. A Resource Guide to the U.S. Foreign Corrupt Practices Act. Disponível em: <https://www.sec.gov/spotlight/fcpa/fcpa-resource-guide.pdf>. Acesso em: 11 de Jul. 2019. p. 3 Historical Background} 

O UKBA foi promulgado com mesmo intuito que a Lei Anticorrupção de 2013 no Brasil, porém a situação que levou o Reino Unido a promulgar esse ato foi a pressão feita pela OCDE. O ato criminaliza qualquer tipo de corrupção, sendo a partir de pessoas físicas ou pessoas jurídicas, é ainda mais ampla que a FCPA, vez que não focou apenas na corrupção estrangeira, mas também na corrupção nacional, envolvendo a Administração Pública dos países e também a corrupção entre particulares. 

Destaca-se que UKBA é se aproxima muito mais da Lei Anticorrupção brasileira, não apenas pela sua abrangência - já que a lei brasileira não pune apenas praticas estrangeiras como E.U.A, como também as nacionais e entre pessoas jurídicas particulares - mas pela importância que a UKBA concede aos programas de integridade. Em seu artigo 9º possui "um guia sobre a prevenção de subornos nas organizações comerciais" com foco em firmar linhas e princípios que serão considerados essenciais para prevenção de suborno na area empresarial. É possível ver a semelhança no caráter preventivo, assim como o programa de integridade brasileiro, ou no mesmo sentido, as praticas de compliance. 

Por fim, a OCDE, obteve fortes influencias sobre o protagonismo dos E.U.A com a promulgação da FCPA em 1997, essa é uma convenção voltada para enfrentar a corrupção em transações comerciais internacionais. Ela determina que os países signatários devam criminalizar expressamente a corrupção de funcionários públicos estrangeiros, bem como houvessem medidas claras para responsabilização da pessoa jurídica que cometesse esse ilícito internacional. 

Assim, o contexto histórico dos grandes atos - ou Act.s - jurídicos que influenciaram o Brasil ao criar a Lei n. 12.846/2013, é possível entender as influencias e os objetivos dos legisladores, as medidas Anticorrupção não surgiram sem um bom estudo e parâmetros anteriores. Portanto, analisando países como o Reino Unido e os Estados Unidos, vemos a força dessas medidas contra suborno e do programa de integridade empresarial. 

\section{Da Natureza da Responsabilidade prevista na Lei Anticorrupção}

A Lei Anticorrupção estabelece a responsabilização objetiva e subjetiva da pessoa jurídica envolvida em casos de corrupção, entretanto, existem divergências doutrinarias sobre o que significam, exatamente, a responsabilização no âmbito cível e no âmbito administrativo. 

Podem ser distinguidos, claramente, três posicionamentos sobre essa responsabilização, sendo: (i) argumentos sobre essa ser uma responsabilidade criminal, defendida por \cite{modestocarvalhosa}; (ii)a responsabilidade com natureza cível sustentada por \cite{JosePimenta}; (iii) e natureza da responsabilidade administrativa sustentada por \cite{MariaSylvia};  

Para que possamos construir um raciocínio jurídico adequado, começaremos recusando a ideia da responsabilidade civil. O Professor José Roberto Pimenta justifica seu posicionamento através do art. 927 do Código Civil de 2002. \cite[Art. 927: aquele que, por ato ilícito, causar dano a outrem, fica obrigado a repara-lo]{Codigocivil}. Deste artigo, surge a responsabilidade objetiva e subjetiva do CC/02. 

A responsabilidade objetiva, portanto, surge de um ato ilícito que casou dano a alguém, e para ser provado ao juízo para que essa pessoa possa ressarcir os danos causados independe-se de aferição de dolo ou culpa. Já a responsabilidade subjetiva, também decorrente de um ato ilícito que prejudicou alguém e essa pessoa deve ter os danos sofridos reparados, todavia, para comprovação e ressarcimento dos danos é necessário provar o dolo ou culpa no ato praticado por essa pessoa. 

Já a corrente que visa caracterizar a Lei Anticorrupção como parte do direito administrativo, fundamenta seus argumentos e posicionamento através de, como diz \cite{MariaSylvia} o "interesses qualificados como próprios da coletividade - internos ao setor público - não se encontram à livre disposição de quem quer que seja, por inapropriáveis. O próprio órgão administrativo que os representa não tem disponibilidade sobre eles, no sentido de que lhe incumbe apenas cura-los - o que é também um dever - na estrita conformidade do que dispuser o intention legis." \footnote{OLIVEIRA, José Roberto Pimenta. Contentarias ao art. 2. In Di Pietro, Maria Sylvia Zanella; MARRARA, thiago (Coodrs). Lei Anticorrupção comentada. Belo Horizonte: Fórum, 2018. p. 24-47.}

Sendo assim, as duas posições, tanto civil quando administrativa são limitadas, não veem a corrupção globalizada como realmente deve-se ver, elas se limitam a respectivamente, a corrupção que causa um dano expresso a alguém que deve ser reparado de alguma forma ou apenas a corrupção que envolva um ente administrativo e assim envolvendo a area que estuda e aplica o direito na administração pública deste país. Porém, como veremos adiante, a corrupção vai além e deve ser responsabilizada as partes independente de um dano concreto causado a alguém ou de seu envolvimento com a gestão pública. 

Neste sentido, a posição adotada por esse texto é que a Lei prevê a responsabilidade criminal no corpo da norma, e essa afirmação se da através dos seguinte fundamentos: (i) o artigo 5º da LAC aproxima-se de tipos penais que estão previstos na legislação penal; (ii) o processo administrativo esta de fato interligado a corrupção dos funcionários públicos, todavia, trata-se de "processo penal-administrativo" afirma o professor Modesto Carvalhosa \cite{modestocarvalhosa}; por fim, (iii) a norma possui sem duvida matéria penal, com formas e efeitos que sancionam aqueles que infringem a materia, sobre nítida restrição de direitos. 

Posicionando a natureza da responsabilidade objetiva e subjetiva prevista na lei, como natureza criminal, pelos motivos supracitados, segue adiante aos demais pontos do trabalho. 

\chapter{Programa de Integridade e o Decreto nº 8.420/2015}

Após analise da natureza da Lei Anticorrupção e pontuando de suas principais influências e demais características consideradas relevantes para a compreensão geral deste trabalho, é necessário adentrar mais ao tema, estudando o programa de integridade associado com o Decreto n. 8.430/2015. 

\section{Compliance ou Programa de Integridade}

O programa de integridade previsto na LAC, como ja demonstrado, sofreu grande influência da FCPA e UKBA, e dentro dessas normas o termo utilizado para denominar o departamento que abre esse canal de denuncia e investigações sobre possíveis corrupções  dentro das empresas é o "Compliance", de origem do verbo "to comply", verbo que significa "obedecer uma ordem". Portanto, nossos legisladores se basearam nesse mesmo departamento de Compliance, porém mudaram o nome para "Programa de Integridade", assim, ambos significam, em essencia, o mesmo, podendo ao decorrer do trabalho tratar o instituto de ambas as maneiras. 

Neste sentido, instituído que Programa de Integridade e Compliance se referem a mesma coisa, as mesmas atitudes que uma empresa deve tomar, juntamos as informações do Capitulo 2 e associamos que o Compliance é também um departamento que cuida da postura ética da pessoa jurídica. Assim, o compliance não cuida apenas do cumprimento das regras e leis, mas também deve propor atividades que fortaleçam a ética das pessoas que trabalham dentro das atividades da pessoa jurídica. A visão do órgão é de prevenção, detenção e remediação dos atos abusivos e ilícitos cometidos contra os funcionários públicos nacionais ou estrangeiros. 

Em caráter teórico, quanto a estrutura interna que rege a pessoa jurídica, o compliance não se confunde com a auditoria interna - que exerce atividades aleatórias para avaliar o cumprimento de normas previstas no estatuto e tem um viés voltado ao regimento de finanças e avaliação de lucro - as atividades do compliance são continuas e rotineiras, é a abertura de um canal para os funcionários, suas atividades nunca devem cessar. Também, não se confunde com o departamento jurídico da pessoa jurídica, ja que, o direito e a jurisdição estão voltadas a elaboração de documentos judiciais, para, por exemplo, resolver problemas de consumidor ou cíveis, mas é possível que o departamento jurídico se envolva para dar continuidade a uma denuncia de corrupção feita pela equipe de compliance, entre outros meios de trabalho correlacionado. 

Assim, para sociedades empresariais de capital aberto, as S.A, o compliance não se confunde com a Comissão de Valores Mobiliários (CVM), ainda que esse órgão vise otimizar o desempenho da pessoa jurídica, protegendo todas as partes interessadas na relação de, por exemplo, emissão de um titulo pela companhia e a compra por uma pessoa física, através da transparência de informações da pessoa jurídica, sua função não é investigar totalmente e diariamente a situação interna da sociedade. 

\subsection{Os Princípios da Governança das Sociedades}

A \cite{ocdeprincipios}, publicou uma cartilha com princípios que devem reger a governança corporativa das companhias. 

"(i) assegurar a base para um enquadramento eficaz do governo das sociedades 

(ii) os direitos dos acionistas e funções fundamentais do seu exercício

(iii) o tratamento equitativo dos acionistas

(iv) o papel dos outros sujeitos com interesses relevantes no governo
das sociedades

(v) divulgação de informação e transparência

(vi) as responsabilidades do órgão de administração". \footnote{ORGANIZAÇÃO PARA A COOPERAÇÃO E O DESENVOLVIMENTO ECONÔMICO. Os Princípios da OCDE sobre o governo das sociedades. OCDE, 2004. Disponível em: <https://www.oecd.org/daf/ca/corporategovernanceprinciples/33931148.pdf>. Acesso em: 13 de Jul. 2019}

Já no artigo 3º do Decreto n.9.203/2017, este que dispõe exclusivamente da política de governança da Administração Públíca, informa-nos sobre os princípios regentes dessa relação: 

"(i) capacidade de resposta;

(ii) integridade;

(iii) confiabilidade;

(iv) melhoria regulatória;

(v) prestação de contas e responsabilidade; e

(vi) transparência".

Esses pilares, instituídos pelo decreto tem como objetivo portanto a boa governança dentro da Administração Publica Estadual. A capacidade de resposta é um desses pontos que deve ser entendido como princípio, poís assim é possível que haja o fortalecimento, significando que todos os funcionários públicos terão conhecimento suficiente para se posicionar independente da situação em que se encontram. Quanto a integridade, esta é uma qualidade que esperamos não só do poder público, como de todas as sociedades empresariais ou simples que se relacionam com pessoas físicas como utilizadores do serviço ou produto ou como funcionários e empregados, assim como a confiabilidade desses, que se confere com continuidade de seus deveres perante a todos. 

O princípio da melhora regulatória e da prestação de conta e responsabilidades, estão interligados entre si, visando o aprimoramento diário da Administração Pública, que devera seguir com suas obrigações de prestar conta para ser sempre transparente com seus gastos, relações e demais responsabilidades que possua de relevância importância para sociedades empresariais e para a população. 

Assim, concluindo os princípios basilares da governança corporativa, existentes tanto no decreto n. 9.203/2017 como na cartilha publicada pela OCDE, que esta diretamente ligada com a Lei Anticorrupção. 

\subsection{A Society of Corporate Compliance and Ethics SCCE}

A cultura do compliance esta se implementando cada vez mais em todas as empresas, isso vem crescendo e uma das provas são as instituições que estão sendo criadas para cuidar e monitorar este assunto. Assim, a Society of Corporate Compliance and Ethics (SCCE) - tradução livre: "sociedade corporativa de compliance e ética" - foi criada em 2004 pelos Estados Unidos da America por diversos profissionais que pautam suas discussões, trabalhos e cursos de formações sobre o tema de ética e compliance \cite{ComplianceEthicsProgram}. \footnote{MURPHY, Josepy E. A Compliance and Ethics Program on a Dollar a Day.Disponível em: <https://assets.corporatecompliance.org/Portals/1/PDF/Resources/CEProgramDollarADay-Murphy.pdf>. Acesso em: 20 de Julho de 2019}

Nos EUA, existem as diretrizes de sentenças federais "Federal Sentencing Guidelines" \footnote{UNITED STATES OF AMERICA. United States Sentencing Comission. Guidelines Manual 2016. Disponível em: <https://www.ussc.gov/sites/default/files/pdf/guidelines-manual/2016/GLMFull.pdf>. Acesso em 20 de Julho de 2019} \cite{Guidelines}. Estas diretrizes sao guias de posições que devem seguir ao determinar suas sentenças, para, por exemplo, tratar das punições das orgânicos corruptas, assim como previsto no capítulo 8, estabelecendo um calculo de multa e outras penalidades. 

Interessante observar que estão elencados 6 (seis) itens que devem ser observados ao julgar uma pessoas jurídica acusada, sendo eles, previsto na \cite{Guidelines} p. 536, Commentary 2., - tradução livre - "(i) o envolvimento com atividades criminosas ou a sua tolerâncias em relação e elas; (ii) o histórico da organização, (iii) a violação de uma ordem; por fim, (iv) a obstrução da justiça" para determinar os fatores que reduzem a pena dessas sociedades criminosas "(i) a existência do cumprimento efetivo do programa de ética e (ii) a autorrenúncia, a cooperação ou a assunção de responsabilidade".

Neste sentido, podemos observar que o Brasil esta caminhando em direção satisfatória, porém, até diante às diferenças dos sistemas Jurídicos entre os EUA e o Brasil, não podemos nos inspirar em tudo e trazer um sistema tão eficiênte e similar para investigação e comparação das empresas, mas a Lei n. 8420/2015 é a base de julgamento das sociedades consideradas corruptas, assim como a lei estabelece parâmetros para suas punições e responsabilidades.

\section{A Controladoria Geral da União (CGU) e seu envolvimento com o setor de Compliance}

Em 2015, a Controladoria Geral da União publicou diretrizes para sociedades empresariais privadas, neste texto \cite{CGUProgramadeIntegridade} podemos observar se refere ao programa de integridade como "um programa de compliance especifico para prevenção, detecção e remediação dos atos lesivos previstos na Lei 12.846/2013", no qual seu objetivo esta voltado a evitar execuções fraudulentas de contratos realizados com a Administração Publica.

Nesta mesma diretrizes, ainda que tenhamos afirmado que o programa de integridade e o compliance façam referencia sobre as mesmas praticas e tenham a finalidade, a Controladoria Geral da União os diferencia; para eles o compliance é um departamento que assegura o cumprimento das leis em geral, não apenas referente a LAC, já o programa de integridade é formado apenas para evitar praticas tipificadas na Lei Anticorrupção, sendo mais especifico quando ao seu foco para lutar contra praticas abusivas e corruptas. Posiciona-se neste momento como uma mera divergência entre os termos, já que observando a linha do tempo e as influências externas que esta lei teve. A posição de que ambos possuem a mesma característica, de combater corrupção e servir como um canal aberto as pessoas que trabalham dentro da pessoa jurídica independendo do tipo de sociedade que esta forma, continua firme. \footnote{BRASIL. Ministério da Transparência e Controladoria Geral da União. Programa de Integridade: diretrizes para empresas privadas. Brasilia. Página 6. Disponível em: <https://www.cgu.gov.br/Publicacoes/etica-e-integridade/arquivos/programa-de-integridade-diretrizes-para-empresas-privadas.pdf>. Acesso em 21 de Julho de 2019.}

 Conclui-se neste tópico a simples e breve divergência terminológica adotada pela CGU e a adotada neste texto acadêmico. Para fins de esclarecimento, continua-se com a posição de que ambos os termos se referem as mesmas atitudes que as sociedades empresariais ou simples devem tomar, portanto, ambas as nomenclaturas serão adotadas e deve-se entender que fazem referencia a estas praticas anticorrupção. 

\section{Dos comentários direcionados ao Decreto n. 8.420/2015}

 A Lei Anticorrupção (LAC ou Lei n. 12.846/2013), não taxa os mecanismos e quais deverão ser os procedimentos internos adotados, breve apenas em seu artigo 7º, caput "Serão levados em consideração na aplicação das sanções", inciso VIII "a existência de mecanismos e procedimentos internos de integridade auditoria e incentivo à denúncia de irregularidades e a aplicação efetiva de códigos de ética e de conduta no âmbito da pessoa jurídica". Além disto, a parte mais importante esta presente no paragrafo único, que prevê "os parâmetros de avaliação de mecanismos e procedimentos previstos no inciso VIII do caput serão estabelecidos em regulamento do poder executivo federal. 

 Assim, desta parte final, do paragrafo único do artigo 7º supracitado, foi extraído a essencialidade do Decreto nº 8.420/2015 que sera analisado adiante. O Cápitulo IV é titulado de "Do programa de Integridade", abrange os artigos 41 e 42, incisos e parágrafos, onde estão contidos parâmetros de avaliação que a Administração Pública deverá ter. Os artigos, vide \cite{DecetoCometarios}, não trazem inovações propriamente ditas, mas trazem os principais parâmetros, do artigo 42, de cada inciso se pode arrancar um princípio do compliance. \footnote{LEQUES, Rosana B. Decreto n. 8.420 define parâmetros para 
programa de integridade. Disponível em: <https://www.conjur.com.br/2015-mar-19/rossana-leques-decreto-define-parametros-programa-integridade>. Acesso em 21 de Julho de 2019.}

O decreto aprofunda-se em pontos em que a Lei Anticorrupção apenas menciona, como por exemplo quais os parâmetros devem ser utilizados para o cálculo da multa; qual sera o processo de responsabilização administrativa adotado, como se dara a celebração do acordo de leniência (acordo este que visa um ajuste entre a Administração Pública e a Pessoa Jurídica que já foi penalizada ou esta em iminência de penalização. Assim, neste acordo, a PJ colabora "entregando" provas e recebe uma diminuição de pena) por fim, estabelecendo essas pormenorizações do programa de integridade, traz os sistemas que devem ser utilizados e difundidos, sendo eles o cadastro de empresas inidôneas e suspensas e o cadastro nacional de empresas punidas, respectivamente CEIS e CNEP. 

As empresas de pequeno porte ou microempresa possuem diferenças entre as empresas já estruturalmente maiores e mais influêntes no mercado, assim, o decreto observando e tendo ciência dessa diferença, § 3º, artigo 42, tem a necessidade de formalidades reduzidas, e ali, há a especificação de itens que não devem ser seguidos, como por exemplo, a analise de riscos periodicamente junto a necessidade de realizar adaptações para efetividade do programa de integridade, ou a existência de canais de denuncia de irregularidades abertos e amplamente divulgado entre os funcionários  e demais denunciantes de boa-fé.

Ainda que a formalidade exigida para empresas de pequeno porte e microempresas, seja considerada reduzida, é necessário que essas demonstrem seu rigor com o programa de integridade de alguma forma, mostrem através de suas possibilidades a implementação e seu comprometimento. O artigo 2º da Portaria Conjunta nº 2.279/2015, criado em atenção ao artigo 2º, § 5º do Decreto nº 8.420/2015, requer a apresentação de relatório de conformidade ou relatório de perfil. Artigo 2º caput: "Para que as medidas de integridade implementadas sejam avaliadas, a microempresa ou a empresa de pequeno porte deverá apresentar", "  I - relatório de perfil" e "II - relatório de conformidade". 

O relatório de perfil, possue suas características no artigo 3º da Portaria Conjunta nº 2.279/2015, para maior precisão de avaliação. Portanto, devem conter areas de atuação, quem são as pessoas responsáveis pela administração da empresa, número de empregados e a estrutura organizacional, por fim, o nível de relacionamento que possue com o setor público, contendo algumas especificações. No artigo 4º da mesma portaria, contém informações sobre os relatórios de conformidade, devendo conter demonstrativos sobre as medidas de integridade adotadas, além disto, como essas medidas contribuíram ara prevenção detecção e remediação dos atos lesivos apurados. No parágrafo único traz um rol exemplificativo de documentos que podem ser anexos a esses relatórios para comprovação das informações concedidas, por exemplo, documentos oficiais, cartas ou declarações. 

Ademais, vale observar que esse tratamento desigual entre as empresas de pequeno porte e micro empresa, respeita o princípio da isonomia, pois essa diferenciação tem como objetivo equiparar empresas de grandes estruturas e as menores. Já que a estruturação e a quantidade de funcionário influênciam muito na rede de compliance. Por exemplo, uma empresa pequena, de familia, em comparação com uma fabrica de algo. A relação entre os funcionários e seus chefes e a estrutura hierárquica se diferencia muito, decerto essa simplificação de formalidades sobre o programa de integridade é um meio de incentivar e garantir a função social da empresa e a possibilidade de implementação de um canal para todos os funcionários e a implementação de um departamento de compliance.

Desta forma, o decreto e a portaria conjunta estudados acima, colocam muitas informações sobre a necessidade de criação deste canal de compliance. Assim, a implementação do programa de integridade é um dever, um onus ou uma obrigação? Há em algum lugar entre as normas citadas e a Lei Anticorrupção que informem a característica de obrigação desta implementação?

A Lei Anticorrupção foi muito influênciada pelas normas internacionais, OCDE, FCPA e UKBA, mas nao tem o mesmo tratamento que estas legislações, já que ao contrário das normas supracitadas, o poder judiciário brasileiro não é outorgado a analise discricionária para o ajuizamento destas ações. Até para a aplicação da sanção administrativa quando cometidos atos considerados ilícitos pela Lei nº 12.846/2013, Decreto nº 8.420/2015 e Portaria Conjunta nº2.279/2015. 

Neste sentido, as autoridades jurídicas brasileiras não estão legalmente autorizadas a optar por não averiguar a tutela do bem jurídico acautelado pela LAC, ainda que o programa de integridade em pelas condições e eficiência. Assim a LAC se difere da FCPA por não conceder as pessoas jurídicas que possuem um departamento de compliance o condão de afastar a responsabilidade da mesma dos crimes praticados, além disto, não autoriza as autoridades competentes a proteção do bem tutelado a analise discricionária 
dia aos atos da previstos na LAC. 

Já, comparado a LAC com a UKBA, para as normas britânicas, o compliance ou programa de integridade representa um dever jurídico, que necessita ser totalmente cumprido, já que esta sob pena de sanção jurídica, assim sendo, o não atendimento da UKBA, no sistema britânico, é um comportamento ilícito das partes envolvidas. Enquanto, novamente, a LAC prevê em seu artigo 7º alguns fatores que deverão ser avaliados antes da sua aplicação absoluta. 

Desta forma, a LAC não impõe um ato ilícito apenas pela não instituição do departamento de compliance, não se caracterizando por um dever jurídico, como explica o professor \cite{RobertoOnus}, seguindo inclusive, desta forma, o princípio da legalidade. O departamento de compliance é um onus da pessoa jurídica, seu dever é apenas não cometer atos ilícitos de corrupção e praticas abusivas, e assim o sistema ofereceu uma sugestão que possibilita maior alcance dentro da pessoa jurídica e controle destas situações que possam surgir. \footnote{GRAU, Eros Roberto. Nota sobre a distinção entre obrigação, dever e ônus. Revista da Faculdade de Direito - Universidade de São Paulo, São Paulo, v. 77, p. 177-183, jan de 1982. Disponível em <https://www.revistas.usp.br/rfdusp/article/view/66950>. Acesso em: 30 jul 2019.}


O onus da aplicação do compliance dentro da empresa é apenas um instrumento pelo qual o sistema jurídico usa para impor o comportamento dentro dos padrões previstos pela Lei Anticorrupção e demais normas que prevem e repudiam essas praticas. Assim, este é  um meio de satisfazer o próprio interesse da sociedade profissional, não interferindo diretamente na aplicação da pena jurídica caso algum ato ilícito tenha sido cometido e julgado pelo poder judiciário.

Compreendendo o compliance, ou programa de integridade, como ônus da pessoa jurídica que deseja se manter dentro da legalidade prevista pelo sistema jurídico brasileiro, lutando contra praticas corruptas enraizadas, esta é apenas um parâmetro a ser seguido para aqueles que desejam, não uma obrigação ou atenuante para os processos judiciais. Desta forma, no capitulo sequente sera tratado a estrutura deste departamento ou programa a ser instaurado. 


\chapter{Estrutura do Compliance}

A Lei Anticorrupção não ofereceu, em seu texto normativo, elementos concretos para constituir o compliance empresarial, temos apenas a sua finalidade que é de prevenir e deter praticas de atos ilícitos. Já o Decreto nº 8.240/2015, descreve mecanismos e procedimentos internos que devem fazer parte do programa de integridade, 