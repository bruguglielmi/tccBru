%!TEX root = index.tex
\chapter{A Lei n. 12.846/ 2013}

Trataremos, primeiramente, da base sobre o instituto de Compliance, a Lei n. 12.846 - sera tratada como LAC - , criada em 1 de agosto de 2013. Esta Lei é denominada como "Lei Anticorrupção"ou "Lei da Empresa Limpa". Conforme seu artigo 1, parágrafo único, aplica-se a todas as socidades empresariais ou sociedades simples, ou seja, personificadas ou não, independente de fatores como sua organização ou modelo societário adotado. Esse foi um significante marco de enfrentamento da corrupção, especialmente ao estabelecer limites a responsabilização objetiva, civil e administrativa, da pessoa jurídica em razão de atos praticados contra a Administração Publica Estatal, nacional ou estrangeira. 

A norma analisada pune, também, na modalidade subjetiva, os dirigentes e administradores da pessoa jurídica infratora (artigo 3, §2 da LAC), isso porque, acredita-se que as pessoas de alto escalão dentro da empresa deve ter conhecimento sobre tudo que ali acontece, tendo consciencia de operações que visam a corrupção para beneficio próprio. Além disso, a Lei não isenta a responsabilização da pesoa jurídica caso o ato seja praticado por terceiro (art. 2 da LAC), tampouco caso haja alteração contratual, transformação ou incorporação, cisão ou fusão societária (art. 4 da LAC), estabelecendo, assim, regras e limites claros para que haja a devida punição objetiva da pessoa jurídica envolvida no esquema de corrupção. 

Alem disto, à previsão para responsabilização objetiva da pessoa jurídica, trazendo o legislador inovação ao prever que para a aplicação das sanções, haverá a avaliação quanto ao eventual programa de integridade no âmbito da pessoa jurídica, definindo na norma como "mecanismos e procedimentos internos de integridade, auditoria e a aplicação efetiva de códigos de etica e conduta" (Art. 7, VIII da LAC)

Assim, na medida em que o legislador estabelece a responsabilidade objetiva contra a pessoa jurídica, a qual o prescinde da avaliação de dolo ou culpa na ocorrência do ato lesivo, a norma indica também o programa de integridade como um instrumento a favor da pessoa jurídica, dando mais importância ao departamento do compliance, que esta totalmente ligado com o programa de integridade que deve existir nas sociedades simples ou empresariais, conforme será demonstrado. Diante da mesma Lei, seja no enfrentamento dos atos de corrupção em suas atividades empresariais e, portanto, de impedimento da incidência da Lei Anticorrupção, seja na eventual fase de dosimetria de sua sanção. 

As características previstas nesta lei, sobre o programa de integridade da pessoa jurídica foi diretamente influenciada pelo contexto jurídico internacional dos Estados Unidos da América, cujos reflexos cumilnaram na celebração de relevantes acordos internacionais de enfrentamento da corrupção, assinadas e internalizadas pelo Brasil. 

Neste sentido , podemos afirmar que a Lei Anticorrupção é o eficaz produto de pelo menos três convenções internacionais de que o Brasil assinou, e se comprometeu a dar eficacia, sendo eles os atos da FCPA - Foreign Corrupt Practicies Act, a UKBA - United Kingdom Bribery Act, e por fim a CFPETI - Convenção sobre o Combate da Corrupção de Funcionários Públicos Estrangeiros em Transações Comerciais Internacionais aprovada pela OCDE. A consonância com o contexto internacional da Lei Anticorrupção, pode representar um grande marco jurídico, ainda mais por incentivar e reafirmar os padrões da etica empresarial, internamente e com as relações negociais perante a Administração Pública Estatal. 

\section{Programa de Integridade}

O programa de Integridade, presente na Lei n. 12.846/2013 em seu art. 7, VIII, foi implementado para trazer novas perspectivas as sociedades empresariais de modo genérico.  O programa deve se ater a três niveis de relações: (i) relações internas; (ii) relações externas com outras pessoas jurídicas; e por fim, (iii) relações externas com o Estado.

As empresas precisam, a partir da vigência dessa Lei, de um canal que poderá ser utilizados por todos os empregados para realizarem denuncias de comportamentos corruptos em todos os niveis hierarquicos internos e externos. Além disso, é necessario que haja um grupo de pessoas que não se submeta a hierarquia dentro da pessoa jurídica, para que eles tenham plenos poderes de investigação e combate a praticas abusivas e corruptas, vindo de qualquer pessoa, seja qual for seu grau de poder dentro do desenvolvimento das atividades empresariais, acusando e detendo, portanto, quaisquer praticas consideradas abusivas perante esta norma. 

Ademais, como a pessoa jurídica nada mais é que uma entidade formada por na verdade, diversas pessoas fisicas que ali atuam e presam por essa entidade em que trabalham, a Lei Anticorrupção prevê que sejam criadas medidas e procedimentos que garantem a possibilidade de identificação, denuncia e aplicação de metodos de coibição mediante os atos de corrupção repudiados pelas politicas da empresa e pela lei em questão. 

Como meio de complemento a Lei Anticorrupção, o Decreto n. 8.420/2015, nos seus artigos 41 e 42, exemplifica itens que devem fazer parte do programa de integridade para que ele seja valido. Como mera exemplificação, o inciso primeiro do artigo 40 deste decreto vale destaque: "I- Comprometimento da alta direção da pessoa jurídica, incluindo os conselhos, evidenciado pelo apoio visivel e inequívoco do programa". Ou seja, como breve explicação, as pessoas com mais poder na empresa devem mostrar a todos os funcionarios que elas tambem estao sujeitas ao programa e todos devem respeita-lo. 

Este instituto traz consigo uma carga de preocupação quanto a sua pratica real. Ainda que haja expressa previsão da retirada desse programa de integridade da escala hierarquica da empresa, é certo a um orgão de alto escalão empresarial não queira e nao goste da ideia de se submeter a outro orgão que nao aqueles expressamente superiores. Como isso, ainda que a norma e o decreto, em forma conjunta, estabeleçam um grande ponto de partida ao combate da corrupção, temos um problema pratico e real quanto a eficiencia desta medida. 

Ainda assim, a obrigação das sociedades de pequeno ou grande porte é de criar um canal onde seus funcionarios possam fazer denuncias de irregularidades. Ademias, as pessoas que gerenciam esse canal devem ter poder para investigar essas denuncias a fundo para fazer com que elas sessem se confirmadas. Sendo assim, observado materias sobre grandes empresas, como JBL, (COLOCAR MAIS UMAS PREPRESSA AQUI E FAZER REFERENCIA NO RODA PÉ DE ALGUMA DESSAS MATERIAS), esse Programa de Integridade é atualmente um orgão/ departamento, que trata da area de compliance da empresa, portanto, o programa nada mais é que o compliance aplicado nas empresas instaladas no Brasil. 

Veja que a ligação é direta, a importancia do Compliance Empresarial diante a Lei Anticurripção de 2013 é clara, sendo que esta preve a criação desse canal, que foi aplicada com a criação de um departamento exclusivo para tratar desses assuntos e analise de niveis de corelações internas e externas. O departamento funciona como um comite de ética, com independência em relação a todos os outros orgãos da pessoa jurídica. 

O departamento de Compliance surgiu, portanto, da necessidade da fiscalização geral da empresa, da necessidade da criação de um canal aberto, onde todos se submetem e podem fazer suas denincias de forma segura, na tentativa da criação de uma empresa solida e sem abusos de poderes e praticas corruptas de qualquer natureza. Assim, devido a complexidade desse orgão, devido as suas multiplas funções, a criação e implementação de um departamento separado se faz necessaria e essencial para essas pessoas juridicas. 


\section{Das Convenções Internacionais assinadas pelo Brasil contra as praticas de corrupção empresarial.}

Como forma de breve esclarecimento, a titulo da menção feita acima, esclarece-se o que são as convenções assinadas pelo Brasil na tentativa de combater e dar mais força a sua repudia como Estado sobre empresas que se utilizam de praticas abusicas e ilegais para obtenção de lucro e lavagem de dinheiro, entre outras possiveis praticas para beneficio ilícito da pessoa jurídica. 

O FCPA (Foreing Corrupt Practices Act), foi um ato promulgado pelo Congresso Norte-Americano em 1977, também conhecida como Lei Contra Praticas de Corrupção Estrangeira. A promulgação foi um marco para o enfrentamento da corrupção corporativa internacional nos Estados Unidos da America. Essa norma responsabiliza civile criminalmente pessoas fisicas e jurídicas que cometem suborno ao poder público estrangeiro. Proíbe também a "contabilidade off-the-books", que significa a proíbição de omissão de valores e recursos utilizados dos livros de registros coorporativos da pessoa jurídica.

Após a norma ser assinada pelo Congresso, devido a grande turbulencia no cenario americano com diversas explosões de casos como "Watergate" e "Lockheed", as investigações se aprofundaram e foi identificada um rede de corrupção global, forçando os Estados Unidos a transformar tal norma em um Tratado Internacional que foi assinado pelos membros da OCDE (Organização para a Cooperação e Desenvolvimento Económico). \footnote{UNITED STATES OF AMERICA. Criminal Division of the U.S Department of Justice; Enforcement Division of the U.S Secritiries and Exchange Comission. A Resource Guide to the U.S. Foreign Corrupt Practices Act. Disponivel em: <https://www.sec.gov/spotlight/fcpa/fcpa-resource-guide.pdf>. Acesso em: 11 de Jul. 2019. p. 3 Historical Background} 

De acordo com \cite{blank2011embrace} a bruna deveria dar um beijo no christian


\section{Da Responsabilidade Objetiva e Subjetiva prevista na Lei n. 12.846/2013 imputada as Pessoas Jurídicas }



\subsection{Da Responsabilidade Civel Objetiva das Pessoas Jurídicas pela LAC}



\subsection{Das Possiveis Sanções e dos Processos Administrativos de Responsabilização dos Aministradores e Diretores}



\subsection{Do Processo Judicial de Responsabilidade dos Administradores e Diretores}



\chapter{Complince - JBL pós Lava Jato }



\chapter{Problematização e Resultado}



