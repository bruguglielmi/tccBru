%!TEX root = index.tex
\chapter{A Lei n. 12.846/ 2013}

Trataremos, primeiramente, da base sobre esse latente assunto, a Lei n. 12.846, criada em 1 de agosto de 2013. Esta Lei é denominada como "Lei Anticorrupção" ou "Lei da Empresa Limpa". Conforme seu artigo 1, paragráfo único, aplica-se a todas as socidades empresariais ou sociedades simokes, ou seja, personificadas ou não, independente de fatores como sua organização ou modelo societário adotado. Esse foi um significante marco de enfrentamento da corrupção, especialmente ao estabelecer limites a responsabilização objetiva, civil e administrativa, da pessoa jurídica em razão de atos praticados contra a Administração Publica estatal, nacional ou estrangeira. 

A norma analisada pune, também, na modalidade subjetiva, os dirigentes e administradores da pessoa jurídica infratora (artigo 3, §2), isso porque, acredita-se que as pessoas de alto escalão dentro da empresa deve ter conhecimento sobre tudo que ali acontece, tendo consciencia de operações que visam a corrupção para beneficio proprio. Além disso, a Lei não isenta a responsabilização da pesoa jurídica caso o ato seja praticado por terceiro (art. 2), tampouco caso haja alteração contratual, transformação ou incorporação, cisão ou fusão societária (art. 4), estabelecendo, assim, regras e limites claros para que haja a devida punição objetiva da pessoa jurídica envolvida no esquema de corrupção. 

Alem disto, à previsão para responsabilização objetiva da pessoa jurídica, trazendo o legislador inovação ao prever que para a aplicação das sanções, haverá a avaliação quanto ao eventual programa de integridade no âmbito da pessoa juridica, definindo na norma como "mecanismos e procedimentos internos de integridade, auditoria e a aplicação efetiva de códigos de etica e conduta" (Art. 7, VIII)

Assim, na medida em que o legislador estabelece a responsabilidade objetiva contra a pessoa jurídica, a qual o prescinde da avaliação de dolo ou culpa na ocorrência do ato lesivo, a norma indica também o programa de integridade como um instrumento a favor da pessoa jurídica, dando mais importância ao departamento do compliance, que esta totalmente ligado com o programa de integridade que deve existir nas sociedades simples ou empresariais, conforme será demonstrado. Diante da mesma Lei, seja no enfrentamento dos atos de corrupção em suas atividades empresariais e, portanto, de impedimento da incidência da Lei Anticorrupção, seja na eventual fase de dosimetria de sua sanção. 



\chapter{Complince - JBL pós Lava Jato }



\chapter{Problematização e Resultado}



