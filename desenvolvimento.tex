%!TEX root = index.tex
\chapter{A Lei n. 12.846/ 2013}

Trataremos, primeiramente, da base sobre esse latente assunto, a Lei n. 12.846, criada em 1 de agosto de 2013. Esta Lei é denominada como "Lei Anticorrupção" ou "Lei da Empresa Limpa". Conforme seu artigo 1, paragráfo único, aplica-se a todas as socidades empresariais ou sociedades simokes, ou seja, personificadas ou não, independente de fatores como sua organização ou modelo societário adotado. Esse foi um significante marco de enfrentamento da corrupção, especialmente ao estabelecer limites a responsabilização objetiva, civil e administrativa, da pessoa jurídica em razão de atos praticados contra a Administração Publica estatal, nacional ou estrangeira. 

A norma analisada pune, também, na modalidade subjetiva, os dirigentes e administradores da pessoa jurídica infratora (artigo 3, §2), isso porque, acredita-se que as pessoas de alto escalão dentro da empresa deve ter conhecimento sobre tudo que ali acontece, tendo consciencia de operações que visam a corrupção para beneficio proprio. Além disso, a Lei não isenta a responsabilização da pesoa jurídica caso o ato seja praticado por terceiro (art. 2), tampouco caso haja alteração contratual, transformação ou incorporação, cisão ou fusão societária (art. 4), estabelecendo, assim, regras e limites claros para que haja a devida punição objetiva da pessoa jurídica envolvida no esquema de corrupção. 

Alem disto, à previsão para responsabilização objetiva da pessoa jurídica, trazendo o legislador inovação ao prever que para a aplicação das sanções, haverá a avaliação quanto ao eventual programa de integridade no âmbito da pessoa juridica, definindo na norma como "mecanismos e procedimentos internos de integridade, auditoria e a aplicação efetiva de códigos de etica e conduta" (Art. 7, VIII)

Assim, na medida em que o legislador estabelece a responsabilidade objetiva contra a pessoa jurídica, a qual o prescinde da avaliação de dolo ou culpa na ocorrência do ato lesivo, a norma indica também o programa de integridade como um instrumento a favor da pessoa jurídica, dando mais importância ao departamento do compliance, que esta totalmente ligado com o programa de integridade que deve existir nas sociedades simples ou empresariais, conforme será demonstrado. Diante da mesma Lei, seja no enfrentamento dos atos de corrupção em suas atividades empresariais e, portanto, de impedimento da incidência da Lei Anticorrupção, seja na eventual fase de dosimetria de sua sanção. 

As caracteristicas previstas nesta lei, sobre o programa de integridade da pessoa jurídica foi diretamente influênciada pelo contexto jurídico internacional dos Estados Unidos da América, cujos reflexos cumilnaram na celebração de relevantes acordos internacionais de enfrentamento da corrupção, assinadas e internalizadas pelo Brasil. 

Neste sentido , podemos afirmar que a Lei Anticorrupção é o eficaz produto de pelo menos duas convenções internacionais de que o Brasil assinou, e se comprometeu a dar eficacia, sendo eles os atos da FCPA - Foreign Corrupt Practicies Act, a UKBA - United Kingdom Bribery Act. A consonância com o contexto internacional da Lei Anticorrupção, pode representar um grande marco jurídico, ainda mais por incentivar e reafirmar os padrões da etica empresarial, internamente e com as relações negociais perante a Administração Pûblica Estatal. 

\section{Programa de Integridade}

O programa de Integridade, presente na Lei n. 12.846/2013 em seu art. 7, VIII, foi implementado para trazer novas perspectivas as sociedades empresariais de modo genérico. 

As empresas precisam, a partir da vigencia dessa Lei, de um departamento que servirá como canal de prevenção e repressão de condutas consideradas lesivas na forma desta lei. Esse programa possui três niveis: (i) relações internas; (ii) relações externas com outras pessoas jurídicas; e por fim, (iii) relações externas com o Estado. 

Ademais, como a pessoa jurídica nada mais é que uma entidade formada por na verdade, diversas pessoas fisicas que ali atuam e presam por essa entidade em que trabalham, a Lei Anticorrupção prevê que sejam criadas medidas e procedimentos que garantem a possibilidade de identificação, denuncia e aplicação de metodos de coibição mediante os atos de corrupção repudiados pelas politicas da empresa e pela lei em questão. 

Como meio de complemento a Lei Anticorrupção, o Decreto n. 8.420/2015, nos seus artigos 41 e 42, exemplifica itens que devem fazer parte do programa de integridade para que ele seja valido. Como mera exemplificação, alguns dos itens são: I- Comprometimento da alta direção da pessoa jurídica, incluindo os conselhos, evidenciado pelo apoio visivel e inequívoco do programa. Ou seja, como breve explicação, as pessoas com mais poder na empresa devem mostrar a todos os funcionarios que elas tambem estao sujeitas ao programa e todos devem respeita-lo, o que traz tambem uma carga de preocupação na pratica real, é certo a um orgão de alto escalão empresarial se submeter a um orgão inferior? Como isso realmente tem sua aplicação na pratica?

Desta feita, a obrigação das sociedades de pequeno ou grande porte é de criar um canal onde seus funcionarios possam fazer denuncias de irregularidades. Ademias, as pessoas que trabalham nesse departamento devem ter poder para investigar essas denuncias a fundo para fazer com que elas sessem se confirmadas, sendo assim, outro nome para esse departamento seria o departamento de compliance, mostrando aqui a ligação direta da importancia do Compliance Empresarial diante a Lei Anticurripção de 2013, este é um comite de etica, com independencia em relação a todos os outros orgãos da pessoa jurídica. 

O departamento de Compliance surgiu, portanto, da necessidade da fiscalização geral da empresa, da necessidade da criação de um canal aberto, onde todos se submetem e podem fazer suas denincias de forma segura, na tentativa da criação de uma empresa solida e sem abusos de poderes e praticas corruptas de qualquer natureza. Assim, devido a complexidade desse orgão, devido as suas multiplas funções, a criação e implementação de um departamento separado se faz necessaria e essencial para essas pessoas juridicas. 


\chapter{Complince - JBL pós Lava Jato }



\chapter{Problematização e Resultado}



