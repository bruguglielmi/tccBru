%!TEX root = index.tex
\chapter{Conclusão}

\section{Discussão} % (fold)
\label{sec:discussao}
Por que a Beeconnect não foi para frente? O que deu errado?

Desde o princípio a ideia que norteou a Beeconnect foi o uso da tecnologia de beacons. Os membros do board da TM foram à Mobile World Conference em 2014 e ficaram encantados com essa tecnologia, que prometia alta precisão de localização indoor. Eles queriam utilizá-los mas não sabiam como. O problema é que esse processo de desenvolvimento da startup pulou uma série de etapas importantes para a geração e desenvolvimento do modelo de negócio. O beacon no caso deveria ser um meio e não um fim.

Pelo fato de ser originada a partir da RM, a Beeconnect ficou com a ideia muito rígida de que propaganda era o caminho a ser trilhado. Por esse fato a equipe perdeu no início cerca de um mês desenvolvendo algo que no final provou ser inútil. Faltou benchmarking e o entendimento do mercado.

Durante o desenvolvimento do aplicativo iShop, que posteriormente foi renomeado para Beeconnect, a equipe ficou muito distante do consumidor final. Faltou coleta de feedbacks e iteração em cima deles. Com tais dados ficaria muito mais fácil de debater em cima do produto ao invés de divagar sobre o que o cliente iria gostar.

Faltou criatividade para a validação da hipótese do aplicativo. Ao invés de gastar seis meses quase quinhentos mil reais em equipe para desenvolvimento seria muito mais simples desenvolver um vídeo, um simples site e algum dinheiro em publicidade no Facebook para que algumas hipóteses fossem validadas rapidamente.

O modelo de negócio era totalmente diferente dos outros modelos da holding. A BCFG é B2C, oferece seus produtos para consumidores, a RM é B2B, oferece produtos para empresas, assim como a BL, enquanto a Beeconnect é B2B2C, oferece produtos tanto para consumidores quanto para empresas. A única sinergia era quanto ao conhecimento em desenvolvimento de servidores. Apesar da BCFG ter desenvolvido uma série de jogos para Android e iOS os desenvolvedores desses jogos não estavam mais presentes então não houve compartilhamento de conhecimento para a programação de aplicativos nessas plataformas.

Um dos maiores problemas para a Beeconnect foi ter que lidar com as duas pontas, as lojas físicas e os usuários dos aplicativos. Modelos de negócio desse tipo demandam muito capital porque para ele funcionar bem seriam necessários muitos usuários e muitas lojas, o que demanda dinheiro e muito tempo. Para que a empresa tivesse uma sobrevida provavelmente a captação de investimento seria a saída, o empecilho é que a holding não gosta de investimentos externos por problemas passados.

Quanto a organização da empresa houve também diversos erros. Foram contratados mais funcionários do que poderia ser absorvido. Isso ocasionou um transtorno no processo de desenvolvimento pois a pessoa que mais produzia tinha que parar para ensinar três pessoas. Outro erro foi ter criado a startup como uma Sociedade em Conta de Participação. Basicamente, quem saiu da RM e foi para a Beeconnect abriu mão de ações da RM para ter a promessa de ganhar mais ações da Beeconnect. Isso ocasionou alguns problemas como falta de empatia por parte das outras pessoas da holding que não viam tanto valor em ajudar a nova startup visto que eles não ganhariam tanta participação no processo. Além disso, essa estrutura era mais rígida e não permitia mudanças rápidas de recursos entre as empresas do grupo TM.

A falta de experiência tanto na programação quanto na parte de negócios foi também um grande empecilho para o sucesso da empresa. Ninguém da equipe tinha criado uma conta de desenvolvedor na Apple. Esse processo é demorado e custoso. Além disso, a empresa de Cupertino é muito rigorosa no processo de avaliação de seus aplicativos. O app de iOS atrasou em quase 2 meses por tais motivos. A cada mês mais e mais capital era drenado da empresa então o quanto antes o MVP fosse lançado melhor seria para iterar sobre o processo de Construir-Medir-Aprender. Outro problema foi a falta de conhecimento na hora de criar um produto novo. Ao chamar o app de iShop e fazer todo o design em cima dele para depois ter que modificar para Beeconnect tomou um tempo considerável. Tal erro custou para a equipe cerca de duas semanas, fora os honorários do advogado.

Outro erro foi ter escolhido o primeiro parceiro para o piloto. O Mc Donalds de Riviera estava muito longe do escritório da empresa. O acompanhamento do piloto é de vital importância para que a coleta de feedback seja feita da maneira correta. A distância não permitiu que a equipe estivesse presente para observar o uso do aplicativo e iterar em cima dele rapidamente.

A startup basicamente falhou por não ter seguido os princípios da Startup Enxuta e do Desenvolvimento do Cliente. Tais princípios teriam poupado muito tempo de programação. O tempo é primordial para uma empresa cujos recursos humanos e monetários são escassos. Outro ponto fundamental é a experiência da equipe. Não havia um programador experiente para Android e iOS, desta forma gastou-se um tempo razoavelmente considerável para o desenvolvimento do aplicativo para essas duas plataformas. A política de contratação também foi falha, faltou uma busca por pessoas comprometidas, no decorrer da empresa saíram seis pessoas, o que é gigantesco para uma startup que teve no seu auge 12 pessoas, isso causa uma queda na motivação e em parte perda de conhecimento, o que é extremamente prejudicial em uma empresa nascente que deve sempre buscar estar motivada e aprendendo cada vez mais. E por fim, ter buscado um modelo negócio cujo tipo não satisfazia com o modelo que a holding busca foi totalmente inadimissível, essa falta de comunicação causou um transtorno que poderia ter sido evitado desde o início.

\section{Lições Aprendidas}
\label{sec:licoes_aprendidas}

À TM fica o aprendizado de uma empresa que falhou por falta de: conhecimento da literatura, política de contratação, experiência da equipe e políticas contratuais sinérgicas para os membros da holding. Além disso, tais aprendizados devem ser repassados de forma humilde para os demais funcionários da holding para que todos saibam as dificuldades enfrentadas e como não errar novamente.

Para o autor ficou aprendizado de que somente a programação e qualidade de desenvolvimento não são suficientes para que uma empresa tenha um produto de sucesso. É necessário muito mais que isso, a literatura desenvolvida durante o trabalho de formatura deve ditar o que as linhas de código devem executar. Isso tornará o trabalho muito mais ágil e fácil de ser iterado para a construção de um produto melhor e que as pessoas queiram utilizar.

Para a USP/Escola Politécnica fica a importância do fomento de criação/incubação de startups. A recomendação é permitir que os jovens arrisquem e aprendam dentro de um ambiente mais seguro. Iniciativas como o Inovalab e o Núcleo de Empreendedorismo da USP permitem um contato maior com startups e abrem novos horizontes para os universitários. Errar é um grande aprendizado, mas o mais importante de tudo é errar o quanto antes. Se os alunos conseguirem criar empresas dentro da universidade como ocorrem nas grandes instituições americanas como Stanford, MIT e Harvard será um grande passo para o futuro do empreendedorismo no Brasil.

Para o mundo fica a lição de que arriscar e buscar os sonhos é fundamental para uma vida empreendedora. É necessário falhar para aprender bastante, é durante os momentos ruins que a reflexão vem do âmago do coração empreendedor. E através dessa reflexão em conjunto com o estudo na literatura que saem novas ideias que possívelmente terão sucesso e mudarão o mundo para melhor.
