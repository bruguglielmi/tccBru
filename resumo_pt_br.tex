%!TEX root = index.tex
% resumo em português
\setlength{\absparsep}{18pt} % ajusta o espaçamento dos parágrafos do resumo
\begin{resumo}

Este é um trabalho de Conclusão de Curso de Graduação da Faculdade de Direito de São Bernardo do Campo, escrito pela Aluna Bruna Guglielmi Pereira no ano de 2019. 

O ponto principal deste trabalho é primeiramente, apresentar o compliance/programa de integridade empresarial, dissertando sobre a Lei que deu origem a esta pratica, Lei nº 12.846/2013 (Lei Anticorrupção), bem como uma breve apresentação do Decreto nº 8.420/2015. A Lei anticorrupção foi inspirada em grandes normas internacionais que graças aos nossos legisladores hoje se aplicam no Brasil para que se de fim as praticas abusivas e corruptas no nosso sistema particular entre empresas e publico administrativo regulando esta interação. 

Assim, é ressaltada a estrutura do compliance alguns de seus nuances e discutido ao final quais são os problemas práticos de sua aplicação que não encontramos solução apenas com a utilização das normas reguladoras, doutrinas e princípios, baseando-se em artigos jornalísticos sobre grandes casos da atualidade. 

\textbf{Palavras-chaves}: Compliance, Direito Empresarial, Lei n. 12846/2013, Lei Anticorrupção.
\end{resumo}